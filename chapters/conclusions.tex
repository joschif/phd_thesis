\thispagestyle{plain}
\section{Conclusions \& Outlook}
\markboth{Conclusions \& Outlook}{}

In the beginning of this thesis, I argued that the convergence of key technological advances allow us to shed a new light on the gene regulatory mechanisms underlying human brain development:

\begin{enumerate}
    \item Brain organoids, which model human brain development in controlled culture environments 
    \item Single-cell genomic profiling techniques, which allow phenotyping biological systems at unprecedented throughput and resolution
    \item Genetic or chemical manipulation in organoids paired with rich multi-omic readouts as an ideal setting for resolving gene regulatory relationships.
\end{enumerate}

Throughout the remainder of this work, we presented new methods and approaches to harness these technologies and reveal regulatory features of brain development. In this chapter, I will summarize how the results of individual chapters fit together and position this work in the broader research context. Finally, I will provide an outlook discussing the limitations of this work and how it could shape the directions of future research.

\subsection{Conclusions}

In Chapter 2, I presented VoxHunt, a computational toolkit to explore and assess organoid regional identities. This addresses a major challenge in the field, as organoid heterogeneity can vary widely between stem cell lines or separate batches. Reliable ways to annotate organoid cell states are therefore crucial to develop new and more robust organoid protocols, and to characterize effects of perturbation experiments. VoxHunt addresses this by facilitating systematic comparisons to large, well-annotated reference datasets of the human and the mouse brain. The computational toolkit was implemented with the goal in mind to make it quick and flexible to use for researchers even with little computational experience. Next to enabling annotation of scRNA-seq datasets, it can estimate organoid region composition through reference-based deconvolution of bulk transcriptomic measurements. This makes VoxHunt a powerful tool to find culture conditions that steer organoid development towards distinct regional identities. We also demonstrate this in the paper by performing a proof-of-principle patterning screen on developing organoids. VoxHunt is now routinely used in our lab (e.g. \cite{kanton_organoid_2019,he_lineage_2022}) and in all subsequent chapters in this thesis. It has also already been successfully used in other studies to annotate mouse cell populations (\cite{la_manno_molecular_2020}) and to validate a novel organoid protocol (\cite{miura_generation_2020}).

Since the publication, we have further extended VoxHunt with new functionalities and we hope that it will be a useful contribution for propelling organoid and stem cell engineering.

Having developed a method to reliably characterize the endpoints of regional diversification, we next wanted to identify the time window and regulatory events of regional bifurcation during organoid development. To this end, we recorded a single-cell multi-omic timecourse over early brain organoid development, which I presented in Chapter 3. This revealed that regionalization in organoids - not unlike in the human brain - happens hierarchically. First, an anterior-posterior axis is set up and forebrain progenitors are established, which later diversify into dorsal and ventral identities. During the time window of forebrain establishment, we also identified and characterized organoid cell populations that were transcriptomically similar to organizers in the mouse brain. Organizer-like populations in organoids were observed previously (\cite{renner_self-organized_2017}), and our work now provides a comprehensive transcriptomic and epigenomic characterization of these populations and shows that they reproducibly arise across stem cell lines. It will be interesting to further explore the functional relevance of these populations. 

Well-integrated scRNA-seq and scATAC-seq data over the entire organoid development from pluripotency is also a rich resource to infer the underlying gene regulatory networks. We found that existing approaches for GRN inference that were designed for scRNA-seq data could not take advantage of multimodal datasets, which motivated the development of Pando. Pando builds upon previous approaches for GRN inference (e.g. \cite{aibar_scenic_2017}) and extends them to leverage the multimodal nature of modern single-cell datasets. By explicitly modeling the interactions of TF expression and binding site accessibility, we can exploit covariance relationships between both modalities that were previously unused. Furthermore, it allowed us to individualize the GRN to create cell type-specific networks based on accessibility constraints. Pando proved useful for revealing gene regulatory relationships in brain organoids as presented in Chapter 3 and 4 and was since also applied in other studies to investigate retina organoid development (\cite{wahle_multimodal_2022}) as well as adult neurogenesis in the axolotl forebrain (\cite{lust_single-cell_2022}).

Since its inception, Pando has been continuously extended with new functionality and an improved interface. Due to the flexible framework it builds on, it can make use of many types of measurements and prior knowledge to enhance GRN inference. For instance, chromatin conformation data or histone modifications as measured by scCut\&Tag were so far not utilized for GRN inference but could add valuable information. Given the ever-expanding repertoire of single-cell profiling methods, this flexibility is important to continue improving the quality of inferred GRNs.

% Possibly expand on limitations of GRN inference -> outlook

Next to providing a comprehensive characterization of gene regulatory events during early organoid development, we sought to probe the functional role of TFs and other transcriptional regulators during this process. To this end, I presented two separate pooled genetic perturbation screens with single-cell transcriptomics readout in this thesis. The first, presented in Chapter 3, assessed requirement of 20 cortical TFs for regional diversification. In the second, presented in Chapter 4, we screened 36 high-risk ASD genes to examine their roles in cell type establishment. Interestingly, in both screens the predominant effect of many perturbations (with some exceptions) was an enrichment in ventral telencephalon identities and a - sometimes complete - depletion of cortical neuron populations. This might well be a byproduct of the target selection. After all, cortically expressed TFs are more likely involved in cortex formation and excitatory/inhibitory neuron imbalance has been described previously for ASD mouse models (\cite{sudhof_neuroligins_2008,nelson_excitatoryinhibitory_2015}). Nevertheless, this observation may be interesting to consider in future studies.

The TF perturbation screen presented in Chapter 3 suggested GLI3 as an important regulator of forebrain dorsalization. We followed this up using GLI3-KO stem cell lines to validate GLI3's requirement for cortex establishment and to better understand the underlying regulatory mechanisms. We found that the transcriptomic changes induced by GLI3 loss-of-function were largely consistent with the inferred GLI3 targets in the GRN. Motivated by this, we used the observed DE results to further refine the GRN graph and propose a multiphasic regulatory model for human telencephalon development. While this model was entirely derived from our data gathered in human organoids, it was consistent with previous studies in mice and additionally proposed novel downstream factors. This demonstrates that organoids recapitulate gene regulatory programs observed in primary tissues and that they can be predictive model systems.

In Chapter 4, we screened ASD risk-genes using a similar paradigm and also used an inferred GRN to better interpret the observed perturbation effects. However, instead of zooming in on one specific regulator, we first assessed broadly whether different ASD risk genes and their perturbation effects converged in certain regulatory subnetworks. Consistent with our expectations, we observed that perturbation effects accumulated in the GRN subgraphs around ASD risk TFs, but additionally found some non-ASD TF modules that were highly enriched for both ASD risk genes and perturbation effects. Finally, we further followed up on ARID1B, a member of the BAF complex. Our screen results suggested that it is involved in ventral telencephalon development, specifically the transition of ventral progenitors to oligodendrocyte precursors. This was further supported by evidence from patient-derived organoids and MRI imaging. 

Taken together, the two studies presented in Chapter 3 and 4 pioneer the application of single-cell genomics and perturbation screens in organoids to systematically resolve the gene regulatory networks at the basis of human brain development. 

In Chapter 5, we sought to extend the study presented in Chapter 3 by exploring the epigenomic landscape of brain organoid development beyond chromatin accessibility. To reconstruct the epigenetic trajectories governing regionalization events, we profiled histone modifications using scCut\&Tag at various stages of brain organoid development. This allowed us to characterize the epigenetic switches at lineage bifurcations that precede and predict expression of cell type-specific genes. We further show that  perturbation of chromatin modifiers destabilizes development and leads to aberrant fate acquisition. In conjunction, Chapters 3 and 5 present a comprehensive transcriptomic and epigenomic characterization of fate establishment during brain organoid development. Moreover, both studies reveal important regulatory factors of development through perturbation experiments.



\subsection{Outlook}

In this last subchapter, I will take a closer look at some central topics of this thesis and discuss their limitations and future perspectives.

\topparagraph{Organoid phenotyping}
Developing new organoid protocols and interpreting the results from perturbation screens are both reliant on robust methods to phenotype cellular heterogeneity. Moreover, quantitatively comparing organoid cell states with primary counterparts is central to assess the precision and accuracy of protocols (\cite{camp_single-cell_2018}). With VoxHunt, we addressed this by mapping expression profiles of organoid cells to primary reference populations. However, quantitatively assessing the fidelity of engineered cell states is still challenging. This is in part because current approaches are limited by the availability and breadth of reference datasets. Particularly reference atlases of human tissues are still limited, but major initiatives, such as the Human Cell Atlas (\cite{regev_human_2017}), are on the way to provide a complete caracterization of all human cell types. Further, efforts to comprehensively survey organoid cell types, such as the Organoid Cell Atlas (\cite{the_human_cell_atlas_biological_network_organoids_organoid_2021}) the will be useful to reveal which cell states have already been generated \textit{in vitro}, and which ones are still missing. Building on this, integrated atlases of primary and organoid tissues may be pivotal, by allowing the comprehensive comparison of cell states across individuals and tissues as well as organoid batches and protocols. Such atlases can also help to train machine learning models with robust uncertainty estimates to quantitatively characterize new engineered cell states. Going beyond transcriptomics, assessing and comparing chromatin state will also be important, for example to understand how stem cell lines are poised to differentiate towards a given brain region. Finally, multi-omics profiling could help quantify and compare "regulatory network states" as a more informative measure of cell identity than individual modalities.


\topparagraph{Regulatory network inference}
In this thesis, I presented Pando, a GRN inference algorithm that leverages multi-omics data and incorporates features of the regulatory genome that have not been previously utilized. In our analysis of GLI3 KO organoids, we found that the GRN inferred by Pando showed strong concordance with the observed transcriptomic effects and was predictive with respect to direct GLI3 targets. Nevertheless, we realized that comprehensive evaluation of GRN-inference performance is extremely challenging due to the lack of viable ground-truth networks. Current benchmarks rely on ChIP-seq data to derive a ground-truth for evaluation (\cite{pratapa_benchmarking_2020}). However, many algorithms (including Pando) explicitly or implicitly use TF motifs derived from ChIP-seq in the inference procedure, which makes this data impractical for unbiased validation. Alternative approaches to evaluate GRNs could be based on how predictive they are for gene expression changes upon TF perturbation. 

There are also still a number of limitations for GRN inference from single-cell transcriptomics and epigenomics data. Approaches such as Pando are based on the assumption that TF and target gene expression are correlated. This is likely not always true, since regulation can occur post-transcriptionally and even on the protein level. Furthermore, Pando relies on motif databases to match TFs with putative binding site and thus the accuracy of motif predictions is a crucial bottleneck. Some recent deep learning-based approaches predict TF binding sites directly from DNA-sequence (\cite{avsec_effective_2021,janssens_decoding_2022}). Such models could be useful to refine binding site predictions or could even be used directly as components of new GRN inference models. In addition, incorporation of other measures of chromatin state such as those obtained from scCut\&Tag could be great ways to annotate active regulatory regions. Using only observational data for GRN inference also makes it challenging to discover causal relationships. In Chapter 3, we have incorporated perturbation signatures \textit{post hoc} to refine the GRN and flesh out the causal interactions. However, with more and bigger perturbation datasets becoming available, it is also feasible to use them directly in the inference step in conjunction with genomic readouts. 

\topparagraph{Perturbation experiments}
Throughout this thesis, we have performed a variety of perturbation experiments in organoids to study functional associations. Mosaic single-cell perturbation screens have allowed us to assess many target genes in parallel. This is very time- and resource-efficient, as it does not require the generation of individual KO lines for each perturbation. However, current technologies have a number of important limitations. First, the mutations that are introduced cannot be read out directly but are assigned through detection of gRNAs. This means that the type of the mutation is not known for a given cell and gRNA efficiency cannot be measured. Second, there is often an imbalance in the number of detected cells for each perturbation, with some perturbations being over-represented in an organoid, while others are detected in only few cells. It is oftentimes not clear if this is due to the natural variation in clonal expansion in organoid culture (\cite{esk_human_2020}), an effect of a given perturbation, or both. The combination of these factors make it often difficult to properly assess perturbation-associated phenotypes and to detect subtle transcriptomic changes. In this thesis, we have addressed some of these issues by controlling for known covariates with appropriate statistical tests and by highlighting effects that were highly consistent across organoids. Additionally, we validated individual associations with KO and patient cell lines. Future implementations of such pooled screens could benefit from the use of new precise base editors (\cite{anzalone_search-and-replace_2019}) as well as the use of clonal barcodes (\cite{he_lineage_2022,esk_human_2020}) to better control for proliferation biases during organoid generation. 

Performing perturbation experiments in a more "classical" arrayed fashion gives greater control over many experimental variables and often allows a more in-depth characterization of the perturbation effects. In this thesis, the experiments conducted in this way usually yielded clearer and more reproducible phenotypic effects than mosaic perturbation screens. While their throughput is currently limited by a number of technical bottlenecks, some emerging technological advances could change this in the near future. Modern cell barcoding methods allow multiplexing many samples and profiling millions of cells in a single experiment (\cite{rosenberg_single-cell_2018,yin_high-throughput_2019,cao_comprehensive_2017}). At the same time, new culture protocols combined with lab automation technology may permit scalable production and treatment of organoids. This would pave the way for massive perturbation screens in organoid systems with information-rich genomic readouts. 


\topparagraph{Towards predictive biology}
An ultimae goal of scientific endeavors like the one presented here is to be able to predict complex phenotypes like fate transitions or responses to stimuli. However, biology is extraordinarily messy and complex, which makes building such predictive models a challenging task.

One can in principle approach this challenge from the "bottom-up" and try to map out all molecular interactions with the ultimate goal of building a dynamical model describing the regulatory landscape. In essence, current approaches for gene regulatory network inference (such as Pando) rely on genomic measurements to learn such interactions between transcription factors and DNA elements. This approach might be greatly enhanced by perturbation screens probing not only transcription factor knockout, but also precise alterations in binding site and protein sequence. At the same time, accurate predictors of protein structure (\cite{baek_accurate_2021,jumper_highly_2021}) and new proteomic technologies may advance our understanding of transcription factor interactions with their binding site and with other transcription factors. In the future, these advances may allow us to annotate all regulatory interations at the molecular level. 

However, this bottom-up approach is inherently limited by the difficulty of dynamically modelling such high-dimensional regulatory networks. Predictive modeling of emerging phenomena might therefore require looking at it from the "top-down". Exposing complex, \textit{in vitro} generated tissues to diverse perturbations and profiling their response using single-cell genomics may allow charting a rich landscape of cellular phenotypes. Using modern machine learning methods, this type of data can be used to learn "paths" between phenotypes directly on this landscape, without the having to model the underlying dynamics. Ultimately, this could facilitate accurate prediction of responses to stimuli and link cell state to fate, which would be transformative for drug discovery and tissue engineering. Taken together, current developments foreshadow an exciting road towards predictive biology.


