\thispagestyle{plain}
\section{Introduction}
\markboth{Introduction}{}


% Figures
% * Organoid technologies overview
% * Layered anatomy of organoid and primary
% * Single cell technologies overview
% * 


The human brain is a remarkably complex organ, made up of a vast diversity of cell types which are assembled into intricate structures. Understanding how its development is orchestrated has long been a central challenge in developmental biology. While studies in humans are challenging due to both ethical and practical reasons, we have already learned a lot about the morphological and molecular factors that shape these processes by examining them in various non-human vertebrates.

During early embryogenesis, a sheet of neuroepithelial cells folds in and pinches off to form the neural tube, which gives rise to most of the central nervous system. Through a spatially and temporally precise cascade of morphogen gradients, the tube is segmented into distinct regions that will eventually develop into forebrain, midbrain, hindbrain and spinal cord. Each of these gradients is induced by a discrete group of adjacent cells, so-called organizers, which express and secret a defined set of morphogens. Most knowledge about this process has been gained from knockout and transplantation experiments in model organisms such as mice, with bulk genomic measurements or tissue stainings with a limited set of markers genes as a readout. As these conventional methods are limited in their throughput and resolution, it has long remained challenging to finely dissect differentiation events and underlying genetic circuits. Further, there are strong differences between the human and the mouse brain with respect to cellular composition, size, morphology and function, and it is unclear whether insights from mouse studies can be transferred to human brain development. Two emerging technologies are now transforming the field and offer exciting new possibilities to study human brain development at unprecedented resolution: ​\textit{In vitro} brain models grown from human stem cells and single-cell genomics.

In this thesis, we want to harness and combine these new technologies to take a fresh approach to the study of human brain development. In particular, we are interested in shedding light on the genetic circuits that coordinate the diversification of brain regions in humans. For this, we first develop new computational tools that help us to characterize regional identities in brain organoids (Chapter 2) and to infer gene regulatory relationships from modern genomic measurements (Chapter 3). Next, we apply these tools (and others) to better understand gene regulation underlying forebrain development in healthy tissues (Chapter 3 \& 5) and under genetic perturbation of risk genes for Autism Spectrum Disorder (Chapter 2). Before delving into these results, however, I will provide a primer on the technological advances that this work builds upon.

\clearpage


\subsection{Brain organoids}

Given the right culturing conditions, human stem cells can self-organize into complex 3D tissues, so called organoids, which can mimic the microanatomy and functionality of various organs, including the brain. The ability to grow such tissues in controlled culture environments has revolutionized the study of human biology and disease in recent years. This is especially true for the brain, where primary tissue is hard to obtain and which is virtually inaccessible to genetic manipulation. The following section should serve as an overview of current approaches to culture brain organoids and discuss strengths and limitations of organoids in modeling human brain development.


\subsubsection{Brain organoid technologies}
Brain organoid generation typically starts with aggregation of stem cells into spheroids or embryoid bodies (EBs). EBs are then cultured in a neural induction medium, which pushes cells towards a neuroectodermal fate. After the neuroectoderm is established, the tissue further self-organizes to form lumens lined with neural progenitor cells, so-called neural rosettes. The emergence of this neuroepithelial stage is aided by embedding in matrigel, a gel containing extracellular matrix components (\cite{lancaster_cerebral_2013,eiraku_self-organizing_2011}). Progenitor cells proliferate and can eventually differentiate into various neuronal and glial cell types that make up the mature organoid. Culturing in certain specialized media can further boost neuron differentiation and specification of neuronal subtypes (\cite{bardy_neuronal_2015}). 

Without the addition of external patterning factors, organoid development is not directed towards any brain-regional identity. Interestingly, organoids grown with such unguided protocols nevertheless contain cell populations resembling multiple distinct brain regions such as forebrain, midbrain or hindbrain (\cite{lancaster_cerebral_2013,kadoshima_self-organization_2013}). This offers the potential to study how different parts of the brain self-organize and interact during development. Understanding the molecular factors governing theses self-patterning mechanisms could be informative about the gene-regulatory basis of neurodevelopmental fate decisions in humans. However, there are disadvantages to this strategy: Regional composition can vary widely between different stem cell lines or even batches grown separately (\cite{kanton_organoid_2019}) and predicting it \textit{a priori} is currently not possible. This can be especially problematic if modeling a disease requires the generation of a specific brain region.

An alternative strategy to grow brain organoids are guided protocols, which use signalling molecules to control patterning of the neuroepithelium to form a defined brain structure. These protocols take inspiration from  developmental biology and supply molecules interacting with known pattering pathways in the mammalian brain (reviewed in \cite{chiaradia_brain_2020}). For instance, WNTs and BMPs dorsalize the neural tube (\cite{dickinson_dorsalization_1995,saint-jeannet_regulation_1997}) and modulating these pathways has been successfully used in a number of studies to generate organoids resembling dorsal forebrain structures (\cite{xiang_fusion_2017,pasca_functional_2015,qian_brain-region-specific_2016,qian_sliced_2020}). Using this guided paradigm, protocols have already been developed to generate various brain structures, including cortex (\cite{eiraku_self-organizing_2011,kadoshima_self-organization_2013,pasca_functional_2015,velasco_individual_2019}), ventral forebrain (\cite{miura_generation_2020,birey_assembly_2017,xiang_fusion_2017,bagley_fused_2017}), hypothalamus (\cite{qian_brain-region-specific_2016}), thalamus (\cite{xiang_hesc-derived_2019}), midbrain (\cite{qian_brain-region-specific_2016,monzel_derivation_2017,kim_modeling_2019}) and others. 

Furthermore, there are approaches to expose the tissue to a spatial gradient of one or a combination of morphogens to control the induction of multiple brain structures in a single organoid (\cite{rifes_modeling_2020,cederquist_specification_2019}). This can also be achieved by fusing organoids resembling one or more brain structures into so-called assembloids (\cite{bagley_fused_2017,birey_assembly_2017,xiang_fusion_2017}, reviewed in \cite{pasca_rise_2018}), which can be very useful for studying neuronal migration and formation of inter-region neuronal connections. This set of guided techniques might enable reproducible generation of brain structures and modeling region-specific diseases.



\subsubsection{Organoids model brain development \textit{in vitro}}

Brain organoids have been shown to recapitulate certain aspects of human brain development, such as characteristic microanatomy and cell type diversity. Even without the addition of external patterning factors, they can give rise to organizer-like populations (\cite{renner_self-organized_2017}), and eventually form distinct brain region identities. The tissue morphology of these brain regions can be very similar to their primary counterpart: Like in the fetal human cortex, the neural rosettes of the organoid neuroepithelium begin to develop a layered architecture along an apical-basal axis (\cite{kadoshima_self-organization_2013,lancaster_cerebral_2013}). Radial glia (RG) cells line the basal ventricular zone (VZ) and differentiate into immediate progenitor cells (IPs), occupying the sub-ventricular zone (SVZ). IPs further differentiate into neurons, which migrate along extended processes of RGs to populate the apical layers. Younger organoids primarily give rise to deep layer-like neurons, but after around 4 months of culture in optimized conditions, more diverse neuronal populations can emerge and stratify into deep and upper layers (\cite{kanton_organoid_2019,qian_sliced_2020}). However, some aspects of this tissue architecture are to date still lacking in organoids. For instance, the fetal cortex has a distinct layer of RG cells in the outer SVZ, so-called outer radial glia cells (oRG). While oRG marker genes are expressed in some organoid cells, they lack the distinct spatial separation of this layer (\cite{bhaduri_cell_2020}). Optimizing oxygenation and nutrient diffusion as well as longer culturing times could be beneficial for more reliable development of cortical layers in organoids (\cite{bhaduri_cell_2020,chiaradia_brain_2020}).

Generally speaking, cell types arising in organoids resemble primary counterparts to a remarkable degree. Both unguided and guided protocols can give rise glutamatergic (excitatory) and GABAergig (inhibitory) neurons with transcriptomic signatures resembling various distinct brain structures. As outlined above, dorsal forebrain-derived excitatory neurons can resemble different cortical layers (\cite{kanton_organoid_2019,qian_sliced_2020}), but they can be also derived from other brain structures, such as diencephalon, midbrain or hindbrain (\cite{kanton_organoid_2019}). For instance, Cajal-Rezius cells, a class of diencephalon-derived excitatory neurons, have been identified in multiple guided and unguided organoid protocols (\cite{kanton_organoid_2019,velasco_individual_2019}). Inhibitory neurons in the primary human cortex are born in the ganglionic eminences (GEs) located in the ventral forebrain and migrate into the cortex during development. Such GE-derived inhibitory neurons can also be found in organoids, both in unguided protocols (\cite{kanton_organoid_2019}) and guided protocols patterned for general or specifically ventral forebrain (\cite{velasco_individual_2019,birey_assembly_2017,miura_generation_2020}). In some cases, they even possess distinct regional transcriptomic signatures of lateral-, medial- or caudal GE (\cite{kanton_organoid_2019,miura_generation_2020}). 

In addition to neurons, the human brain also contains various glial cell types that are crucial for brain function. In organoids, such glial cells can also arise, especially in later stages of development. Astrocytes, which perform many functions in the brain, can be very abundant in brain organoids and even overgrow neurons in very long cultures ($\geq$ 4 months) (\cite{kanton_organoid_2019,giandomenico_cerebral_2019,sloan_human_2017}). Moreover, oligodendrocytes and oligodendrocyte progenitors, which go on to create the myelin sheath around neural axons, can be produced from organoids (\cite{tanaka_synthetic_2020}). Their induction can also be specifically induced through alterations in the protocol (\cite{madhavan_induction_2018,marton_differentiation_2019}). Such protocols are promising, as they might help to better control proportions of certain cell types to better mimic primary tissue.

Despite fast progress in the development of new and refined organoid protocols, there also remain several limitations to brain organoids as model systems. Because all brain organoids to date are derived exclusively from the neuroectodermal lineage, they are inherently missing cells from other germ layers, most importantly mesoderm, which gives rise to microglia and endothelial cells of blood vessels. As a result of lacking vascularization, nutrient and oxygen supply is limited by diffusion, which becomes increasingly insufficient if the organoids grow larger. This leads to a necrotic core especially in older organoids and limited maturation of neurons as mentioned above (\cite{vertesy_cellular_2022,bhaduri_cell_2020}). 

Moreover, the lack of stereotypic patterning often results in high variability between orga-noids grown from different stem cell lines or in separate batches. While the transcriptomic signatures of cell types is generally reproducible, the proportion of different cell types can vary drastically, especially with unguided protocols (\cite{kanton_organoid_2019}). Furthermore, there is so far no comprehensive reference of organoid cell types, which makes it often challenging interpret and annotate data from new organoid protocols. Thus, increasing the reproducibility of organoid culture and developing robust methods for rapid phenotyping are crucial especially for larger-scale drug screening applications. 



\subsubsection{Applications of brain organoids}
Given the many parallels between human fetal brain development and organoids as well as the human genomic context, it is not surprising that they have quickly emerged as useful model systems for various biological questions that cannot be studied in humans directly. Brain organoids can also readily be genetically and chemically manipulated and even grown from patient cell lines, which gives them unprecedented advantages over animal models. 

Disease modeling is currently one of the most prominent applications of brain organoid technology. Especially brain malformations are well suited to be studied in organoids, due to their early onset and strong disease phenotypes (\cite{klaus_altered_2019,lancaster_cerebral_2013}). More recently, autism spectrum disorder (ASD) has also been studied in organoids by introducing haploinsufficiency for three ASD risk genes (\cite{paulsen_autism_2022}). Furthermore, organoids have successfully been used to model infectious diseases affecting the brain, such as the Zika virus (\cite{qian_brain-region-specific_2016,dang_zika_2016}) and to screen for antiviral compounds (\cite{zhou_high-content_2017}). 

Organoids also offer fresh opportunities to study the genetic mechanisms underlying brain evolution. By growing organoids from primate cell lines like chimpanzee or macaque and comparing them with human organoids, studies have explored evolutionary differences in brain developmental timing and identified human-specific gene expression patterns and regulatory regions (\cite{kanton_organoid_2019,mora-bermudez_differences_2016,pollen_establishing_2019}).

These results demonstrate that organoid systems offer great potential as model systems of the human brain and the combination with genetic editing techniques and chemical screening makes them especially powerful to study mechanisms of human brain development in health and disease.

\clearpage


\subsection{Single-cell genomic technologies}

% First paragraph good, but then too much on atlases. Should probably be more of an aside in the context of the thesis
% - Atlases help to phenotype
% - Instead, focus on development -> multiome -> GRN -> perturbation
% - Then lead into the following sections
% - Better motivation with context to the thesis topic

During development from pluripotency, organoids turn from a relatively homogenous ball of stem cells into heterogeneous tissues with many cell types and diverse regional identities. Therefore, studying organoid development and resolving differentiation events requires probing them on the level of individual cells. Over the past decade, an exciting new set of tools has emerged to perform increasingly complex measurements of various phenotypic features in single cells. Initial methods focused on gene expression by making mRNA-sequencing sensitive enough for the single-cell regime (\cite{tang_mrna-seq_2009}). Since then, the repertoire of single-cell technologies has expanded to allow measurements of various genomic and epigenomic readouts as well as proteins and spatial features. The throughput of these assays has also scaled up rapidly (\cite{svensson_exponential_2018}) and modern commercial solutions now allow profiling up to 1 million cells in a single experiment (\cite{srivatsan_massively_2020,mulqueen_high-content_2021}). 

Advances in measurement technologies are accompanied by the development of computational methods to deal with increasingly large and complex datasets (\cite{zappia_exploring_2018}). Next to tools developed specifically for single-cell genomics, many methods for high-dimensional data analysis from other areas of statistics and machine learning have proven very useful in the single-cell realm. Such modern analytical tools have become essential components of data analysis workflows to perform pre-processing, quality control and sophisticated data interpretation. The combination of high-throughput and high information content assays with advanced computational methods now allows us to take a systems-level view at biology and have already significantly advanced our understanding of biological systems.

These technological advances are critical for studying developmental processes in orga-noids for three main reasons:

\begin{enumerate}
    \item {\scshape Primary references}: The power to characterize molecularly distinct cell types in complex tissues without prior purification has catalyzed many studies surveying composition of primary human and mouse brain tissue (\cite{la_manno_molecular_2021,zeisel_molecular_2018,polioudakis_single-cell_2019,mayer_developmental_2018}). These brain cell atlases are valuable references for understanding organoid development and cell type identities.
    \item {\scshape Developmental trajectories}: Analyzing tissues at single-cell resolution allows us to capture all intermediate cell states between precursors and differentiated cells in a single snapshot. This can reveal contiguous changes in gene expression that occur along the differentiation path and help identify intermediate cell states. 
    \item {\scshape Regulatory inference}: Integrating single-cell transcriptomic and epigenomic measurements opens up new opportunities for inferring gene regulatory relationships. Additionally, genetic perturbation screens with single-cell genomic readout can further elucidate causal relationships.
  \end{enumerate}

In the following sections, I will describe how different genomic modalities can be measured on a single-cell level and outline how we can combine these modalities to better understand gene regulation. Finally, I will discuss how perturbation experiments can help reveal causal relationships and how they can be employed in a high-throuput manner with single-cell readout.


\subsubsection{Single-cell transcriptomics}

% What are we massuring?
% History
% Methods and chemistries
% Analysis pipeline best practices

In order for (protein coding) genes to exert their function on the cell, they first need to be transcribed into messenger RNA (mRNA), spliced, translated into an amino acid sequence and folded into functional proteins. These molecular machines then perform most functions in a cell. Regulating the transcription of genes is one way for a cell control which and how many proteins are active at a given time. Different cell types express distinct subsets of genes, which lets them perform specialized functions. Measuring genome-wide mRNA transcript abundance of a cell (collectively called the transcriptome) using single-cell RNA-sequencing (scRNA-seq) has thus become a useful way to characterize cell types and to provide a proxy for protein expression. 

ScRNA-seq is currently the most mature and most widely used method for single-cell molecular profiling. While there are many protocols and technologies for performing scRNA-seq, they all have some underlying principles in common. Individual cells are first lysed or permeabilized, then their mRNA molecules are reverse transcribed into complementary DNA (cDNA), provided with a unique cell barcode and amplified using polymerase chain reaction (PCR). Most modern protocols also add a unique molecular identifier (UMI; \cite{islam_quantitative_2014}) for each mRNA molecule in order to correct for amplification bias. The amplified cDNA library is further sequenced and finally transcript counts are obtained by aligning sequencing reads to a transcriptome reference. ScRNA-seq protocols majorly differ in how unique cell barcodes are introduced. Most of the earliest methods that were developed are plate-based and isolate individual cells into (micro-)wells, where they are lysed and supplied with individual barcodes (e.g. \cite{picelli_smart-seq2_2013,shalek_single-cell_2014,jaitin_massively_2014,treutlein_reconstructing_2014}). Some of these methods are highly sensitive with respect to transcriptome coverage, but they are limited in the number of cells they can profile in a single experiment. More recent droplet-based methods, such as Drop-seq (\cite{macosko_highly_2015}), inDrops (\cite{klein_droplet_2015}) or the commercial 10X Genomics Chromium platform, use microfluidic chips to encapsulate cells in oil droplets containing barcoded beads. These methods allow profiling thousands of cells in parallel and are currently most widely used in the field. To achieve even higher throughput, recent methods rely on a split-pool-barcoding paradigm and can scale to millions of cells (\cite{rosenberg_single-cell_2018,yin_high-throughput_2019,cao_comprehensive_2017}). Here, the cells are never physically isolated, but a unique combinatorial index is introduced by iterative barcode ligation, pooling and splitting. These compounding technological developments have lead to an exponential increase in throughput of cells over the past decade (\cite{svensson_exponential_2018}).

% All methods outlined above eventually yield sequencing reads that can be aligned to a reference transcriptome and matched with their cellular barcodes and mRNA molecules to obtain a cell-by-transcript count matrix. From there, a number of computational steps is required to arrive at meaningful biological insight. 


% First, the data is quality controlled to remove low quality cells with low count depth and doublets, which typically possess unexpectedly high counts. Next, counts are normalized to correct for count depth bias. Many normalization methods exist (e.g. \cite{lause_analytic_2021,hafemeister_normalization_2019,townes_feature_2019}), but most commonly log-normalization is applied. Here, counts are divided by the total number of counts per cell, multiplied by a scaling factor (counts per million, CPM) and subsequently $log(x+1)$ transformed. 



\subsubsection{Single-cell epigenomics}
% What are we massuring?
% Methods and chemistries
% Analysis pipeline best practices

Transcription is regulated through epigenetic mechanisms, chromatin accessibility, euchromatin, histone marks



\subsubsection{Integrative single-cell multi-omics}
% Integrating different layers
% Methods to meassure omics in parallel
% Possible avenues


\subsubsection{Inference of gene regulatory networks}
% General principles of GRN inference
% Model-based inference methods: SCENIC, CellOracle
% Analysis of GRNs
% Challenges


\subsubsection{Perturbing single cells}
% General principles of perturbations
% * isogenic KO 
% * chemical perturbation
% * single-cell mosaic KO
% Mosaic KO approaches
% Applications 
% Challenges




\subsection{Aim of the thesis}
This project will aim to thoroughly characterize the cell states arising in cerebral organoids and provide a tool for unbiased annotation of cell types, which will be instrumental for the community. Further, I will finely dissect the regulatory logic controlling early cell fate decisions in cerebral organoids by combining single-cell multi-omic measurements with CRISPR/Cas9 activator and repressor perturbation screens. Ultimately, this project will reveal the master regulators of regional cell fate specification during human brain development and shed light on the mechanisms underlying self-organized patterning in cerebral organoids.


\beginbibliography