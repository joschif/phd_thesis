%%%% MAIN TEX FILE FOR THESIS %%%%

\documentclass[12pt, a4paper, twoside]{article}


%%% GENERAL FORMATTING STUFF %%%
%%% not necessary when using xelatex to compile %%%
% \usepackage[utf8]{inputenc}	% UTF8 encoding
\usepackage[T1]{fontenc} % Enables proper hyphenation
\usepackage{scrextend} % Font size stuff
\usepackage{listing} % Source code formatting
\usepackage{layout} % Layouting options
\usepackage{pdfpages} % Enable including full pdf pages


%%% GENERAL LAYOUT %%%
\usepackage[onehalfspacing]{setspace}
% Uses the package 'geometry' to define the format of the page:
\usepackage[
    a4paper, % A4 paper
    left=2.50cm, right=2.50cm, top=2.50cm, bottom=2.50cm, % 2.5 cm margin to the sides
    bindingoffset=10mm, % 10 mm offset for binding
    includeheadfoot % Include header and footer
]{geometry}


%%% SPECIFY FONT %%%
% Set main font (serif font)
% \usepackage{fontspec}\setmainfont{Baskerville}
\usepackage{fontspec}\setmainfont{Adobe Garamond Pro}
% \usepackage{fontspec}\setmainfont{New Caledonia LT Std}

% Set secondary font (sans serif font)
% \setsansfont{Helvetica Neue}
% \setsansfont{Rockwell}
\setsansfont{Futura}

% Defines \code command to switch to monospaced font
\usepackage{seqsplit} % Split long words
\newcommand{\code}[1]{\texttt{\seqsplit{#1}}}


%%% SECTION HEADER FONT %%%
\usepackage{titlesec}
\usepackage{sectsty}
\setcounter{secnumdepth}{4} % Enables numbering for paragraphs
% Section: smallcaps, LARGE, sans serif
\sectionfont{\bfseries\LARGE\sffamily}
% Subsection header: bold, large, sans serif
\subsectionfont{\bfseries\large\sffamily}
% Subsubsection header: bold, normal, sans serif
\subsubsectionfont{\normalfont\large\sffamily}
% Paragraph header: bold, normal, sans serif -> For results and methods of manuscripts
% \topparagraph -> adds linebreak after header (results)
% \paragraph -> no linebreak after header (methods)
\makeatletter
\renewcommand{\paragraph}{\@startsection{paragraph}{5}{\z@}%
  {3.25ex \@plus1ex \@minus.2ex}%
  {-1em}%
  {\normalfont\normalsize\sffamily}}
\newcommand{\topparagraph}[1]{\paragraph{#1}\mbox{}\\}


%%% HEADER & FOOTER %%%
\usepackage{fancyhdr}
\fancyhf{}
% Height of the header
\setlength{\headheight}{15pt}
% Current section name in header
\fancyhead[LE,RO]{\textsc{\leftmark}}
% Pagenumber in footer
\fancyfoot[C]{\thepage}


%%% FIGURES %%%
\usepackage{graphicx} % Allows changing figure sizes
\usepackage{subfigure} % Allows subfigures in figures
\usepackage{float}
\usepackage{ccaption}
\usepackage{caption}
% \usepackage{fltpage} % Allows captions on next page
% Path for all graphics
\graphicspath{ {figures/} }
% Recognizes .pdf and .png automatically
\DeclareGraphicsExtensions{.pdf, .png, .jpg}
% Separator in Label. E.g. Figure 1 | Bladiblubb...
\DeclareCaptionLabelSeparator{line}{ | }
% Font: Helvetica, fontsize 8pt, linehight 12pt
\DeclareCaptionFont{helv}{\mdseries\fontsize{8}{12}\sffamily}
\captionsetup[figure]{font=helv, labelfont=bf, labelsep=line}


%%% TABLES %%%
\usepackage{booktabs} % Better lines in tables
\usepackage{longtable} % Linebreaks in tables
\usepackage{multirow} % Multirow tables
\usepackage{rotating} % Rotate table pages
\usepackage{arydshln} % Draw dashlines
\usepackage{csvsimple} % Include csvs
\usepackage{array}
\captionsetup[table]{font=helv, labelfont=bf, labelsep=line} % Table captions


%%%% OTHER STUFF %%%%
\usepackage{acronym} % Allows to use acronyms
\usepackage{adjustbox} % Draw boxes 
\usepackage{lipsum} % Lorem ipsum text
% \usepackage{authblk} % Author formatting 
% \usepackage{bbding} % Pretty symbols

\usepackage[hidelinks]{hyperref} % Enable hyperlinks ([hidelinks] to disable)


%%% MATH SPECS %%%
% Defines how equations are displayed
\renewcommand{\theequation}{Equation \arabic{equation}}


%%% SCIENCE %%%
\usepackage{dnaseq}	% Nucleotide sequences
\usepackage{texshade} % Aligned sequences 
\usepackage[version=4]{mhchem} % Chemical formulae
\usepackage{siunitx} % SI units
\usepackage{amsmath} % Prettier formulae
\usepackage{listings} % um Code zu setzen


%%% CITATION & BIBLIORGRAPHY STYLE %%%
\usepackage[
    % Management engine
    backend=bibtex,
    % Style: Author, Year. E.g. (R Sanchez 2213)
    style=authoryear-comp,
    citestyle=authoryear-comp,
    bibstyle=authoryear,
    % Sort Citations by year
    sortcites=true,
    sorting=nyt,
    % Max 10 names in bibliography
    maxbibnames=10,
    % Max 2 name in citation
    maxcitenames=2,
    uniquelist=false,
    uniquename=init,
    % Initials
    giveninits=true,
    % DOI, ISBN, URL & eprint not shown in bibliography
    doi=true,
    isbn=false,
    url=false,
    eprint=false
]{biblatex}

% Define command to start bibliography
\newcommand{\beginbibliography}{
    % Add to Table of Contents
    \addcontentsline{toc}{section}{Bibliography}
    \printbibliography
    \clearpage
}

% Path to file exported from mendeley
\addbibresource{bib/thesis.bib}


%%% SUPPLEMENT %%%
% Define command to start Supplement
\newcommand{\beginsupplement}{
    % Set conter of tables and figures to 0
    \setcounter{table}{0}
    \renewcommand{\thetable}{S\arabic{section}.\arabic{table}}
    \setcounter{figure}{0}
    \renewcommand{\thefigure}{S\arabic{section}.\arabic{figure}}
}

%%% CHAPTER %%%
% Define command to start new chapter
\newcommand{\newchapter}[1]{
    % % Set conter of tables and figures to 0
    \setcounter{table}{0}
    \setcounter{figure}{0}
    % Reset S naming for supplement
    \renewcommand{\thefigure}{\arabic{section}.\arabic{figure}}
    \renewcommand{\thetable}{\arabic{section}.\arabic{table}}
    \input{#1}
    \clearpage
}

%%% BLANK PAGE %%%
% Define command to insert blank page
\newcommand{\blankpage}{
    \newpage
    \thispagestyle{empty}
    \
    \newpage
}



%%%% ACTUAL DOCUMENT %%%%
\begin{document}

\begin{titlepage}
\centering

{DISS.\ ETH NO.\ 28706}\\

\vspace{2cm}

{\bfseries\sffamily\LARGE
Exploring brain development in organoids with single-cell multi-omics
}

\vspace{2cm}

{\large A thesis submitted to attain the degree of}\\
{\sc \large Doctor of Sciences} {\large of} {\sc \large ETH Zürich}\\
{\large(Dr.\ sc. ETH Zürich)}\\


\vspace{1cm}

{\large presented by}\\

\vspace{0.3cm}

{\sc \large Jonas Simon Fleck}\\
{\large M.Sc. Ruprecht-Karls-Universität Heidelberg, Germany\\}

\vspace{1cm}

{\large 
born on 26. September 1991\\
citizen of Germany\\
}

\vspace{2cm}

{\large
accepted on the recommendation of\\
Prof.\ Dr.\ Barbara Treutlein, examiner\\
Prof.\ Dr.\ Randall J. Platt, co-examiner\\
Prof.\ Dr.\ Fabian J. Theis, co-examiner\\
}

\vspace{2cm}

{\large
2022
}
	
\end{titlepage}
\clearpage

\pagenumbering{roman}
\begin{center}
    \large\textsc{Abstract}
\end{center}
\markboth{Abstract}{}
\addcontentsline{toc}{section}{Abstract}

During human brain development, cells undergo remarkable fate transitions to give rise to a vast diversity of cell types. This process is tightly coordinated by a complex regulatory landscape that guides cells towards their terminal fate. It has long remained challenging to explore the molecular mechanisms that shape this landscape in humans. The convergence of two new technologies now offers fresh possibilities to study human brain development: \textit{In vitro} brain models grown from human stem cells and single-cell genomics. 

In this thesis, we use brain organoids to model cell fate acquisition during early human brain development and probe them using modern single-cell technologies. In the first part, we address the challenge of systematically characterizing organoid phenotypes. For this, we develop VoxHunt, a novel computational tool to assess the regional composition of brain organoids by mapping them to large reference atlases. In the second part, we illuminate the gene regulatory events shaping brain region diversification. We generate a single-cell transcriptome and chromatin accessibility atlas over a dense time course of early organoid development. We develop Pando, a flexible framework for inferring gene regulatory networks (GRNs) from multi-modal datasets and use it to infer a global GRN underlying human brain organoid development. We apply pooled genetic perturbations to assess transcription factor requirement for brain region diversification and show that the GLI3 is required for cortical fate establishment in humans. In the third part we were curious how brain development is altered in autism spectrum disorder (ASD) and performed a pooled genetic screen of ASD risk genes. We show that perturbations in many ASD genes lead to alterations of cell fate establishment and identify an ASD-specific sub-GRN. Specifically, we find that perturbation of ARID1B leads to aberrant fate transitions towards oligodendrocyte precursor cells. In the last part, we extend our previous multiomic atlas of early human brain development to shed light on epigenetic mechanisms. We profile three histone modifications and mRNA expression in single cells over an organoid developmental time course and identify epigenetic switches controlling cell fate bifurcations. Through perturbation of epigenetic modulators, we further show that epigenetic regulation is required to stabilize fate commitment during early organoid development. 

In summary, we here present new methods and insights to help illuminate the regulatory landscape underlying human brain development. We also provide valuable multi-omic reference atlases of brain organoid development and pioneer the application of perturbation screens in organoids to systematically interrogate gene regulation. Altogether, this thesis provides a framework for how human organoid systems and single-cell technologies can be leveraged to reconstruct human developmental biology.



\clearpage

\begin{center}
    \large\textsc{Acknowledgements}
\end{center}
\markboth{Acknowledgements}{}
\addcontentsline{toc}{section}{Acknowledgements}
Thanks mom!
\clearpage


% \tableofcontents
% \clearpage

% \blankpage
% \includepdf[fitpaper=true, pages=-]{pdfs/chapter_1.pdf}

% \pagestyle{fancy} % Fancy header and footer
% \pagenumbering{arabic} 	

% \newchapter{chapters/introduction.tex}

% \blankpage
% \includepdf[fitpaper=true, pages=-]{pdfs/chapter_2.pdf}

% \newchapter{chapters/voxhunt.tex}

% \blankpage
% \includepdf[fitpaper=true, pages=-]{pdfs/chapter_3.pdf}

% \newchapter{chapters/pando.tex}

% \includepdf[fitpaper=true, pages=-]{pdfs/chapter_4.pdf}

% \newchapter{chapters/asd.tex}

% \blankpage
% \includepdf[fitpaper=true, pages=-]{pdfs/chapter_5.pdf}

% \newchapter{chapters/cnt.tex}

% \blankpage
% \includepdf[fitpaper=true, pages=-]{pdfs/chapter_6.pdf}

% \newchapter{chapters/conclusions.tex}

% \beginbibliography

\end{document}         


